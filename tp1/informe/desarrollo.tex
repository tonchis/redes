\section{Desarrollo}

\indent \indent Para realizar las capturas de paquetes necesarias para la confección de este trabajo práctico se implementó una herramienta en python utilizando las bibliotecas de scapy. En el archivo \textit{sniffer.py} se encontrará el código fuente de la \textit{tool}. Consideramos oportuno mencionar que nuestra herramienta captura, en modo promiscuo, todos los paquetes que puede de la red local para luego volcarlo en un archivo con extensión \textit{.pcap}, soportado por aplicaciones como Wireshark.\\
\indent Si bien es cierto que scapy nos provee la funcionalidad suficiente como para simplemente capturar los ARP necesarios, nos pareció lógico que la herramienta capturase todo lo que pudiese, puesto que filtrar los datos con el Wireshark es bastante fácil de hacer.\\
\indent En el archivo \textit{helpers.py} se podrá encontrar todas las funciones encargadas del cálculo de las entropías de las fuentes origen y destino, así como también el cálculo de la información de cada símbolo de las fuentes. Con ayuda de un intérprete de Python como \textit{iPython} dichas funciones son fácilmente aplicables.\\
\indent Sobre el cálculo de la entropía queríamos comentar ciertas cuestiones. En primer lugar, que en vez de la probabilidad de aparición de cada IP utilizaremos la frecuencia de aparición de dicha dirección como fuente o destino de toda la captura dependiendo del caso. En segundo lugar, que calculamos para cada caso la frecuencia de aparición de todas las IP's, tanto origenes como destino, y luego realizamos el cálculo de la entropía. Notar que si una IP sólo aparece como origen no impactará en el cálculo de la entropía para la fuente de IP's destino y viceversa, puesto que su frecuencia de aparición en dicha fuentes será igual a cero.\\
\indent Intuitivamente, uno imagina que los nodos significativos de cada fuente se corresponden con aquellos que más apariciones hacen en ella, es decir aquellos que más participan como origen o destino en la correspondiente fuente, o equivalentemente los nodos con mayor frecuencia de aparición en cada fuente. Esto, en términos de información se traduce en que un nodo significativo debería proveer menos información que los demás. Siguiendo esta línea de pensamiento, es lógico pensar en los nodos significativos como aquellos cuya información es menor a la entropía de la fuente, considerando a esta última como la media de información de la fuente.\\
\indent En nuestra tool, la función \textit{distinguished\_nodes} devuelve, dado un .pcap con los paquetes ARP capturados, los nodos distinguidos de las fuentes \textit{Fuente} y \textit{Destino}, en concordancia con el criterio expuesto en el párrafo anterior y se puede utilizar como una herramienta para obtener de manera automática los nodos distinguidos de una captura.\\
\indent En base a estos datos se crearon gráficos de barras que contrastan la información de cada símbolo de una fuente con la entropía. Aprovechamos aquí para mencionar que en estos gráficos no se mostrarán todos los símbolos de la fuente, sino los que menos información contengan ordenados ascendentemente, para así facilitar su lectura y análisis.\\
%Para detectar nodos significativos tomamos en cuenta la \textit{información} de cada IP en ambas fuentas. Cuanto mayor es la información de una IP en una fuente, menor es su probabilidad (o en nuestro caso, frecuencia de aparición). Con lo cual, para detectar nodos con mucha participación en la red tenemos que buscar aquellos que menos \textit{información} proveen, en particular, aquellos cuya \textit{información} es menor a la \textit{entropía} de la fuente.\newline 
\indent Para confirmar visualmente los resultados que se obtengan de aplicar el método anterior utilizaremos grafos dirigidos con los datos de las capturas. Para ello, se implementaron dos scripts en Python que se encargan de, dado un archivo \textit{.pcap}, exportar los datos a formatos \textit{.tgf} (Trivial Graph Format) y \textit{.dot} respectivamente, que nos ayudaran para la armado de dichos grafos.\\
