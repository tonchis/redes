\section{Introducción}

\subsection{Address Resolution Protocol}

\indent \indent  Se refiere con \textit{Address Resolution Protocol} (o ARP) a un protocolo utilizado para resolver direcciones de la capa de la red a direcciones de la capa de enlace. Se trata de un protocolo de \textit{request and reply} y sólo se utiliza dentro de los límites de una red.\\
\indent ARP es usualmente puesto en uso para relacionar direcciones IP's con direcciones físicas, como una dirección MAC.\\
\indent Existen dos tipos de paquetes ARP en el protocolo, diferenciados por la función que realizan: 
\begin{itemize}
\item Los paquetes \textbf{who-has} son aquellos paquetes de petición enviados por un dispositivo de forma broadcast a los demás nodos de la red. Su función es, dicho coloquialmente, preguntar cuál es el nodo que posee una determinada dirección IP.
\item Los paquetes \textbf{is-at} son los paquetes de respuesta. Son enviados de manera unicast a quién haya hecho la petición de la dirección IP, informando la dirección MAC correspondiente a dicha IP. 
\end{itemize}

\indent De los distintos componentes de un paquete ARP, nos enfocaremos en los siguientes:
\begin{itemize}
\item \textbf{Operation}: Especifica la operación que se está realizando. 1 para petición y 2 para respuesta.
\item \textbf{Sender Hardware Address}: La dirección MAC del emisor. Para los paquetes de tipo who-has indica la dirección del que haya realizado el pedido. En los paquetes is-at se corresponde con la dirección que se peticionó.
\item \textbf{Sender Protocol Address}: En nuestro caso, la dirección IP del emisor.
\item \textbf{Target Hardware Address}: Ignorado en los paquetes de petición. En los paquetes is-at contiene la dirección MAC del host que realizó el pedido.
\item \textbf{Target Protocol Address}: En nuestro caso, la dirección IP del receptor.
\end{itemize}

\subsection{Entropía}

\indent \indent En teoría de la información se entiende por \textbf{entropía} a la cantidad promedio de información contenida en un mensaje recibido, donde por \textit{mensaje} se refiere a un evento o una muestra de un conjunto de datos. De alguna forma, la entropía muestra nuestra incerteza respecto de de una fuente de información.\\
\indent La idea es que mientras menos probable sea un evento, su ocurrencia proporcionará más información.\\
\indent Basada en un modelo probabilístico, la fórmula cerrada para calcular la entropía de una fuente con k eventos probables es:\\
\\
\indent \indent \indent \indent \indent \indent \indent \indent \indent \indent - $\sum \limits_{i=1}^k p_i * \log{p_i}$ \newline

\indent La unidad de la entropía depende de la base del logartimo elegida para calcularla.\\

\subsection{Sobre el objetivo del trabajo práctico}

\indent \indent En este trabajo práctico se nos requirió realizar una \textit{tool} para capturar paquetes de diversas red. Una vez capturados los paquetes, el objetivo fue calcular las entropías de dos fuentes: conformadas por as direcciones origen de los paquetes ARP who-has, por un lado, y de las direcciones destino de los mismos paquetes por el otro.\\
\indent Finalmente, se nos pidió realizar un análisis con los datos que obtuvimos. Para ello se nos recomendó realizar histogramas y grafos dirigidos donde el nodo origen se corresponde con la dirección emisora y el nodo destino con la dirección pedida en los paquetes ARP who-has.