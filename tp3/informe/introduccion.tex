\section{Introducción}

\subsection{Retransmission Timeout (RTO)}

\indent El \textit{timeout de retransmisión} es un valor utilizado en protocolos 
como \textbf{TCP} que sirve para \textit{asegurarse la entrega} de un paquete al
recipiente, a pesar de la ausencia de todo \textit{feedback} de su parte. Se 
encarga de reenviar los paquetes al no recibir confirmación de recepción.\\

\indent La idea del \textbf{RTO} es, entonces, que dicho valor refleje el tiempo a esperar desde el envío de un paquete hasta el envío de la retransmisión correspondiente en caso de no recibir un reconocimiento del receptor.\\

\indent Intuitivamente es fácil imaginar que un valor óptimo para el \textbf{RTO} es aquel que se aproxime de la mejor manera al \textbf{RTT} promedio de la comunicación, dado que un valor al \textbf{RTT} supondría el envío de una retransmisión antes de la llegada de la respuesta del receptor (en caso de haberse efectivizado), y un valor mayor al \textbf{RTT} supondría una pérdida de tiempo antes de retrasmitir.\\

\indent De esta manera, es de esperar que a medida que se realicen progresivas comunicaciones entre dos host el \textbf{RTO} se vaya recalculando para reflejar así el tiempo recomendado de espera antes de retrasmitir un paquete.\\

\indent Para computar el \textbf{RTO}, el emisor del paquete mantiene dos
variables de estado, \textbf{SRTT} (\textit{smoothed round-trip time}) y \textbf{RTTVAR}(\textit{round-trip time variation}), así como también hace uso de ciertas constantes, $\alpha$, $\beta$ y $K$.\\

\indent El algoritmo básico propuesto por el \textbf{RFC 6298} cuenta con los siguientes pasos:\\
\begin{itemize}
\item Para la primer medición de RTT, fijar:\\
SRTT = RTT\\
RTTVAR = $\frac{RTT}{2}$ \\
RTO = SRTT * max(G, K*RTTVAR), siendo G la granuralidad del reloj y K=4.\\
\item Para subsecuentes mediciones de RTT, debe ejecutarse, en este orden:\\
RTTVAR = (1 - $\beta$) * RTTVAR + $\beta$ * $|$SRTT - RTT $|$\\
SRTT = (1 - $\alpha$) * SRTT + $\alpha$ * RTT\\
RTO = SRTT * max(G, K*RTTVAR), siendo G la granuralidad del reloj y K=4.\\
\end{itemize}

\indent El paper en cuestión sugiere utilizar $\alpha = \frac{1}{4} $ y $\beta = \frac{1}{8}$, valores que suponemos surgen de resultados empíricos en pruebas del algoritmo.\\

\indent Cualquier implementación debe manejar el/los timer(s) de retransmisión
de forma tal que \textit{un segmento \textbf{nunca} es retransmitido demasiado
temprano} (es decir, en menos de un RTO luego de la transmisión del segmento
anterior).\\

\indent A continuación, dejamos el algoritmo \textbf{recomendado} en el
\textbf{RFC 6298} para el manejo del timer de retransmisión:
\begin{enumerate}
 \item Cada vez que se envía un paquete con datos (incluyendo una
	retransmisión), si el timer no está corriendo, se inicia el mismo para
	que expire luego de \textit{RTO} segundos (para el valor actual de
	\textit{RTO}).
 \item Cuando se hizo \textit{ACK} de todos los datos, se apaga el timer de
	retransmisión.
 \item Cuando se recibe un \textit{ACK} que confirma nuevos datos, se reinicia
	el timer de retransmisión para que expire luego de \textit{RTO}
	segundos (para el valor actual de \textit{RTO}).
\end{enumerate}
Cuando expira el timer de retransmisión, hacer lo siguiente:
\begin{enumerate}
 \setcounter{enumi}{3}
 \item Retransmitir el segmento más viejo que no haya sido confirmado
	por el receptor de TCP.
 \item El emisor debe setear \textbf{RTO $\leftarrow$ RTO * 2} (``hacer
	\textit{back off} del timer"). Como valor para acotar superiormente
	esta operación, se puede usar el valor de 60 segundos.
 \item Comenzar el timer de retransmisión para que expire luego de \textit{RTO}
	segundos (para el valor de \textit{RTO} conseguido luego de hacer la
	operación de duplicación en el ítem número 5).
 \item Si expira el timer esperando el \textit{ACK} de un segmento \textit{SYN}
	y la implementación está usando un \textit{RTO} menor a 3 segundos, el
	\textit{RTO} debe ser re-inicializado a 3 segundos cuando comienza la
	transmisión de datos (es decir, luego de finalizar el three-way 
handshake).
\end{enumerate}

\subsection{Sobre este Trabajo Pŕactico}

\indent El objetivo de este trabajo práctico será analizar el comportamiento del protocolo PTC provisto por la cátedra con respecto al cálculo del RTO, simulando en un esquema cliente-servidor en el ámbito de una red local delay y pérdida de paquetes en el envío de los \textbf{ACK} por parte del segundo.\\

\indent Además, se experimentará alterando los valores de $\alpha$ y $\beta$ propuestos por el algoritmo descripto en la sección anterior y se estudiará su impacto a la hora de calcular el RTO, con el fin de determinar algunos valores óptimos que lo acerquen tanto como sea posible al RTT real.