\section{Introducción}

\subsection{Retransmission Timeout (RTO)}

El \textit{timeout de retransmisión} es un valor utilizado en protocolos 
como \textbf{TCP} que sirve para \textit{asegurarse la entrega} de un paquete al
recipiente, a pesar de la ausencia de todo \textit{feedback} de su parte. Se 
encarga de reenviar los paquetes al no recibir confirmación de recepción.\\
\\
\indent Para computar el \textbf{RTO} actual, el emisor del paquete mantiene dos
variables
de estado: \textbf{SRTT} (\textit{smoothed round-trip time}) y \textbf{RTTVAR}
(\textit{round-trip time variation}).\\
\\
\indent Cualquier implementación debe manejar el/los timer(s) de retransmisión
de forma tal que \textit{un segmento \textbf{nunca} es retransmitido demasiado
temprano} (es decir, en menos de un RTO luego de la transmisión del segmento
anterior).\\
\\
\indent A continuación, dejamos el algoritmo \textbf{recomendado} en el
\textbf{RFC 6298} para el manejo del timer de retransmisión:
\begin{enumerate}
 \item Cada vez que se envía un paquete con datos (incluyendo una
	retransmisión), si el timer no está corriendo, se inicia el mismo para
	que expire luego de \textit{RTO} segundos (para el valor actual de
	\textit{RTO}).
 \item Cuando se hizo \textit{ACK} de todos los datos, se apaga el timer de
	retransmisión.
 \item Cuando se recibe un \textit{ACK} que confirma nuevos datos, se reinicia
	el timer de retransmisión para que expire luego de \textit{RTO}
	segundos (para el valor actual de \textit{RTO}).
\end{enumerate}
Cuando expira el timer de retransmisión, hacer lo siguiente:
\begin{enumerate}
 \setcounter{enumi}{3}
 \item Retransmitir el segmento más viejo que no haya sido confirmado
	por el receptor de TCP.
 \item El emisor debe setear \textbf{RTO $\leftarrow$ RTO * 2} (``hacer
	\textit{back off} del timer"). Como valor para acotar superiormente
	esta operación, se puede usar el valor de 60 segundos.
 \item Comenzar el timer de retransmisión para que expire luego de \textit{RTO}
	segundos (para el valor de \textit{RTO} conseguido luego de hacer la
	operación de duplicación en el ítem número 5).
 \item Si expira el timer esperando el \textit{ACK} de un segmento \textit{SYN}
	y la implementación está usando un \textit{RTO} menor a 3 segundos, el
	\textit{RTO} debe ser re-inicializado a 3 segundos cuando comienza la
	transmisión de datos (es decir, luego de finalizar el three-way 
handshake).
\end{enumerate}
