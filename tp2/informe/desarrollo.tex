\section{Desarrollo}

\indent \indent Para la confección de este Trabajo Práctico se implentó una herramienta en \textit{Python} que nos provee la posibilidad de hacer un \textit{traceroute} para un host. \\
\indent Para ello, nos valimos del módulo \textit{Scapy} para realizar los envíos de los paquetes y recibir las respuestas correspondientes.\\
\indent Para cada valor de \textbf{TTL}, enviamos varios paquetes \textbf{ICMP} de tipo \textbf{echo-request} (en una cantidad que puede modificarse por párametro) y calculamos cada \textbf{RTT aproximado}, tomando el tiempo inmediatamente antes y después del envío del paquete y de la llegada de la respuesta. Una vez obtenidos los RTT's de estos paquetes, calculamos su \textbf{promedio} para así obtener un \textbf{RTT representativo} para el correspondiente salto. Luego, siguiendo la lógica explicada en la introducción del trabajo, pero teniendo en cuenta la \textbf{IP} del último de estos paquetes correspondiente a un valor de \textbf{TTL} enviados. Finalmente nos guardamos en una estructura \textbf{la IP con su RTT correspondiente}. Es de notar que de acuerdo a cómo lo calculamos el RTT corresponde al tiempo que tarda en ir el paquete a un host y volver su respuesta. Por lo tanto, para calcular los $RTT_i$ correspondientes a cada salto recurrimos a restar ese valor de RTT obtenido por el del anterior salto identificado.\\
\indent Con esos datos, la herramienta implementada calcula los \textbf{Z scores} correspondientes a cada salto y, finalmente, la \textbf{localización geográfica aproximada} de cada una de las IP's identificadas en la ruta.\\
\indent Una vez implementada la herramienta, procedimos a realizar las experimentaciones correspondientes. Se eligieron las siguiente universidades para tomar mediciones:\\
\begin{itemize}
\item \textbf{University of Oxford}, localizada en el Reino Unido. \url{www.ox.ac.uk}
\item \textbf{University of Tokyo}, sita en Japón. \url{www.u-tokyo.ac.jp}
\item \textbf{Univesity of Auckland}, ubicada en Nueva Zelanda. \url{www.auckland.ac.nz}
\end{itemize}

\indent La elección de estas universidades para realizar las pruebas se debió a que estaban en otros continentes al del host desde el que enviamos los paquetes y así las rutas que estos debían tomar involucraban enlaces submarinos.\\
\indent Dado que todos los integrantes del grupo tuvimos problemas con nuestros proveedores de servicio a la hora de realizar las pruebas (\textit{bloquean} todo tipo de mensaje \textbf{ICMP}, haciendo que ningún nodo responda algo útil para nuestro trabajo práctico), decidimos utilizar una \textit{máquina virtual} alquilada en \textbf{Digital Ocean} para enviar los paquetes. La máquina utilizada se encuentra ubicada en \textit{Nueva York, Estados Unidos}. Todas las experimentaciones se realizaron enviando paquetes desde esta máquina.\\