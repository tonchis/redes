\section{Desarrollo}

\indent \indent Para la confección de este Trabajo Práctico se implentó una herramienta en Python que nos provee la posibilidad de hacer un traceroute para un host. \\
\indent Para ello, nos valimos del módulo Scapy para realizar los envíos de los paquetes y recibir las respuestas correspondientes.\\
\indent Para cada valor de TTL, enviamos varios paquetes ICMP de tipo echo-request (en una cantidad que puede modificarse por párametro) y calculamos cada RTT aproximado, tomando el tiempo inmediatamente antes y después del envío del paquete y de la llegada de la respuesta. Una vez obtenidos los RTT's de estos paquetes, calculamos su promedio para así obtener un RTT significativo para el correspondiente salto. Luego, siguiendo la lógica explicada en la introducción del trabajo, pero teniendo en cuenta la IP del último de estos paquetes correspondiente a un valor de TTL enviados. Finalmente nos guardamos en una estructura la IP con su RTT correspondiente. Es de notar que de acuerdo a cómo lo calculamos el RTT corresponde al tiempo que tarda en ir el paquete a un host y volver su respuesta. Por lo tanto, para calcular los $RTT_i$ correspondientes a cada salto recurrimos a restar ese valor de RTT obtenido por el del anterior salto identificado.\\
\indent Con esos datos, la herramienta implementada calcula los Z scores correspondientes a cada salto y, finalmente, la localización geográfica aproximada de cada una de las IP's identificadas en la ruta.\\
\indent Una vez implementada la herramienta, procedimos a realizar las experimentaciones correspondientes. Se eligieron las siguiente universidades para tomar mediciones:\\
\begin{itemize}
\item \textbf{University of Oxford}, localizada en el Reino Unido. \url{www.ox.ac.uk}
\item \textbf{University of Tokyo}, sita en Japón. \url{www.u-tokyo.ac.jp}
\item \textbf{Univesity of Auckland}, ubicada en Nueva Zelanda. \url{www.auckland.ac.nz}
\end{itemize}

\indent La elección de estas universidades para realizar las pruebas se debió a que estaban en otros continentes al del host desde el que enviamos los paquetes y así las rutas que estos debían tomar involucraban enlaces submarinos.\\
\indent Dado que todos los integrantes del grupo tuvimos problemas con nuestros proveedores de servicio a la hora de realizar las pruebas, decidimos utilizar una máquina virtual alquilada en Digital Ocean para enviar los paquetes. La máquina utilizada se encuentra ubicada en los Estados Unidos. Todas las experimentaciones se realizaron enviando paquetes desde esta máquina.\\